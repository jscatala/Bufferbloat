\section{The Recipe for Bufferbloat}
\subsection{The Transport Layer Protocol}

\begin{frame}
	\frametitle{TCP 101}
	\begin{block}{Packet Conservation Principle}
		\centering
		\textit{``A new packet isn't put into the network until an old packet leaves''.}
	\end{block}
\end{frame}

%\begin{frame}
%	\frametitle{TCP 101}
%	\begin{itemize}
%		\item Three way handshake between them and sequence number.
%		\item TCP makes use of the idea of pipe size and the assumption there was reasonable
%			buffering along the data path.
%		\item Sent data and ACK data $\rightarrow$ Cumulative Scheme.
%		\item Retransmission $\rightarrow$ Adaptative Scheme.
%		\item Receive Window and Congestion Window.
%	\end{itemize}
%\end{frame}

\begin{frame}
	\frametitle{TCP's Phases}
	\begin{block}{}
		\begin{enumerate}
			\item Slow Start
			\begin{itemize}
				\item Exponentially increase of the sending rate.
				\item Increase \textbf{\textit{cwin}} until \textbf{\textit{ssthrsh}}.
			\end{itemize}
		\end{enumerate}
	\end{block}
	\begin{block}{}
		\begin{enumerate}
				\setcounter{enumi}{1}
			\item Congestion Avoidance 
			\begin{itemize}
				\item Optimal utilization preventing congestion.
				\item Most common: Cubic \& CTCP.
			\end{itemize}
		\end{enumerate}
	\end{block}
	\begin{block}{}
		\begin{enumerate}
			\setcounter{enumi}{2}
			\item Fast Retransmit \& Recovery
			\begin{itemize}
				\item Recover from lost without much hurt.
				\item Continuous transmitting of new data on subsequent duplicate acknowledgments.
			\end{itemize}
		\end{enumerate}
	\end{block}
\end{frame}

%%%%%%%%%%%%%%%%%%%%%%%%%%%%%%%%%%%%%
\subsection{Router's Buffers Effect}
\begin{frame}
	\frametitle{Router's Impact}
	\begin{block}{Main Functions}
		\begin{itemize}
			\item To absorb bursts of traffic coming from the hosts.
			\item To ensure that links are used to their maximum capacity.
		\end{itemize}
	\end{block}
	\begin{itemize}
		\item The presence of buffers is necessary to help reduce data loss.
		\item The absence of timely notification triggers the presence of full buffers and increased communication latency.
	\end{itemize}
\end{frame}

\begin{frame}
	\frametitle{Full Buffers}
	\begin{block}{Packets at bottleneck queue}
		The packet is squeezed down in bandwidth but must stretch out in time since its size stays constant.
	\end{block}
	\begin{block}{Standing Queue}
		if  rate $@packets$ are received $>>$  delay to process and remove a package from the buffer.
	\end{block}
	\begin{block}{}
		\centering
		\textit{``This standing queue is the essence of Bufferbloat only creating long delays in communication.''}
	\end{block}
\end{frame}

\begin{frame}
	\frametitle{Other Factors}
	\begin{block}{High Latency}
		\begin{itemize}
			\item Large buffers only increase latency.
			\item It is quite common to find these high-latency queues in the last mile.
		\end{itemize}
	\end{block}

	\begin{block}{Sizing Router's Buffers}
		\begin{itemize}
			\item Unmanaged buffers are more critical today
			\item Buffers sizes are larger, delay-sensitive applications are more prevalent, and large downloads common
			\item Rule-of-Thumb \textbf{\textit{B = RTT*C}}
		\end{itemize}
	\end{block}

\end{frame}
%%%%%%%%%%%%%%%%%%%%%%%%%%%%%%%%%%%%%%%%%
\subsection{Active Queue Management}

\begin{frame}
	\frametitle{Queue management}
	\begin{block}{Tail-Drop Drawbacks}
		\begin{enumerate}
			\item lockout.
			\item full queue that impact with a high queue delay.
		\end{enumerate}
	\end{block}
	
\end{frame}


\begin{frame}
	\frametitle{Queue management}
	\begin{block}{Active Queue Management}
		\centering
		\textit{``FIFO based queue management mechanisms to support end-to-end congestion control''}
	\end{block}

	\begin{block}{Objectives}
		\begin{itemize}
			\item Reduce the average queue length. 
			\item Decrease the end-to-end delay.
			\item Reduce packet losses $\rightarrow$ more efficient resource allocation.
		\end{itemize}
	\end{block}
\end{frame}

\begin{frame}
	\frametitle{Algorithms}
	\begin{block}{}
		\begin{enumerate}
			\item Random Early Detection - RED
			\begin{itemize}
				\item Can provide high throughput and low average delay, but it's complex to configure properly .
			\end{itemize}
			\item BLUE
			\begin{itemize}
				\item Based directly on packet loss and link utilization, can reduce the buffer size and the end-to-end delay.
			\end{itemize}
			\item Control Delay - CoDel
			\begin{itemize}
				\item Based on the idea of a threshold. Beside the minimum packet sojourn, CoDel needs the target and interval time.
			\end{itemize}
		\end{enumerate}
	\end{block}
\end{frame}

%%%%%%%%%%%%%%%%%%%%%%%%%%%%%%%%%%%%%%%%%
\subsection{Hidden Flaws and Bufferbloat}
\begin{frame}
	\frametitle{Evolution of Components}
	\begin{itemize}
		\item Use's Evolution
		\item ``The more is better'' mentallity.
		\item Flows synchronization.
		\item Packet lost seen as a problem.
	\end{itemize}
\end{frame}


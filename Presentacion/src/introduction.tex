\section{Introduction}

\begin{frame}
	\frametitle{The Routers}
	
	\begin{itemize}
		\item The traffic in a network is inherently bursty
			\begin{itemize}
				\item The role of the buffers in the router is to smooth the flow of traffic.
			\end{itemize}
		\item Bottleneck Routers
			\begin{itemize}
				\item Each packet is squeezed down in bandwidth, it must stretch out in time since its size stays constant.
			\end{itemize}
	\end{itemize}
	

\end{frame}

\begin{frame}
	\frametitle{What is Bufferbloat?}
	\begin{itemize}
	\item As stated by \textit{Jim Gettys}
		\begin{itemize}
			\item Today’s networks are suffering from unnecessary latency and poor system performance.
			\item Large buffers damage or defeat the fundamental congestion-avoidance algorithms of the Internet’s most common transport protocol.
		\end{itemize}
	\end{itemize}

	\begin{block}{Bufferbloat}
		\textit{``The existence of excessively large and frequently full buffers inside the network''.}
	\end{block}
\end{frame}

\begin{frame}
	\frametitle{General Objectives}
	\begin{block}{}
		\begin{itemize}
			\item To define the Bufferbloat phenomenon, and explain the impact that it could have on latency and Throughput(related to System Throughput) in Internet.
		\end{itemize}
	\end{block}
	\begin{block}{}
		\begin{itemize}
		\item To detect its presence by measurements of the latency and throughput in a TCP/IP Network.
		\end{itemize}
	\end{block}
	\begin{block}{}
		\begin{itemize}
			\item To propose solutions in the implementation of a network where the existence of excessively large and frequently buffers are detected.
		\end{itemize}
	\end{block}
\end{frame}

\begin{frame}
	\frametitle{Secondary Objectives}
	\begin{block}{}
		\begin{itemize}
			\item Develop appropriate tests to be able to prove the existence of Bufferbloat.
		\end{itemize}
	\end{block}
	\begin{block}{}
		\begin{itemize}
			\item To test and differentiate the possible causes of the excessive latency and throughput reduction in a TCP/IP LAN and check how much is generated by Bufferbloat or by a miss-configuration.
		\end{itemize}
	\end{block}
	\begin{block}{}
		\begin{itemize}
			\item To propose configuration of the TCP parameters in a Linux based machine or an algorithm that can help to minimize the phenomenon.
		\end{itemize}
	\end{block}
\end{frame}

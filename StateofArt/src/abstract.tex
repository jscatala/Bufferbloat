In past years, the growth in the access and appropriate use of that Internet had experienced has been overwhelming. Among the main causes of this fact may be mentioned the increasing use of Internet in our daily lives and the incorporation of Smart Phones and Tablets, with the new ways of use the network due to overcrowding and the creation of new applications such as video and music streaming in high quality, online games and other applications that make heavy use of this resource.
\\

Over time, the value of the hardware has been decreasing (Moore's Law), yet we have created the need for more and more resources for hardware; even when the algorithms governing communications over the Internet have not changed in the same proportion. This, coupled with the proliferation of oversized buffers has been the cause of a new phenomenon in our networks called \textit{Bufferbloat}.
\\

\textit{Bufferbloat} is called the excessively long buffers which spend most of the time saturated, damaging or defeating the fundamental congestion avoidance algorithms of the Internet's most common transport protocols, and that their main task is precisely to avoid excessive congestion and possible loss of our data.
\\

While the problem of \textit{Bufferbloat} is presented in a very specific environment, the theory dictates that it is highly likely to be present on routers that are located as a gateway to the smallest bandwidth segment of the data path; this problem results in that the entire network is affected by the decrease in the data flow.
\\

In this work, the author intend to demonstrate through the study of the theoretical framework of this phenomenon the formation and existence  of the \textit{Bufferbloat} for the TCP protocol, and then with the use of different tools, to test empirically through five different methods under uncontrolled environments, the presence of this phenomenon in home networks with different characteristics.

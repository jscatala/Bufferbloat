Ensure communication only checking the endpoints has revealed that it is not
enough and that it is also necessary to check the status of in-flight data
transmission. With the passage of time and with increasingly high speed
networks it has become more and more important to have mechanisms that keep
throughput high but average queue size low. The changes implemented in TCP for
congestion control have proven not to maximize the capacity of the medium
along with a performance degradation. Among the most common problems are:
connections still experience multiple packet loss, low link utilization and
congestion collapse. It is important to keep in mind that when a packet is
dropped before it reaches its destination, all of the resources it has
consumed in transit are wasted.

This is how the role of routers extends beyond joining and interconnecting local
networks and the Internet together. The role of routers becomes important
because they are able to better manage congestion that may occur in networks
due the knowledge they have about other routers in the network and can choose
the most efficient path for the data to follow, or as already seen, send
message to the ends to fall back or even drop data.  They must allocate
the available bandwidth fairly between all flows.

Typically the queues that routers maintain are designed to smooth and
accommodate bursts of data delivered by the hosts and transient congestion,
but as the queues start to fill, the routers drop packets using a FIFO based
drop-tail management. This discipline has two mayor drawbacks:
\begin{enumerate}
\item Lock Out: a small number of flows tends to monopolize the usage of the
buffer capacity\cite{evolvshortlongflows} .
\item Full Queue: these buffers tends to be always full, leading to 
high latency.
\end{enumerate}

Also, as explained in \cite{FloydJacobsonRED}, when a router use tail-drop,
\emph{the more bursty the traffic from a particular connection, the more
likely it is that the gateway queue will overflow when packets from that
connection arrive at the gateway/router}.

All this led to develop and implement more aggressive or active strategies for
congestion control on the Internet called Active Queue Management or AQM. AQM is a group of FIFO based queue management mechanisms to
support end-to-end congestion control in the Internet. The goals behind AQM
are to reduce the average queue length and with that decreasing the end-to-end
delay. Also to reduce packet losses that reflect as a more efficient resource
allocation.

AQM maintains the network in a region of low delay and high throughput by
dropping packets before queues become full and can reliably distinguish
between propagation delay and persistent queuing delay. Also, because the
router can monitor the size of the queue over time, the gateway is the
appropriate agent to detect incipient congestion. The most common AQM
algorithms are: RED, SRED, BLUE, SFB, CoDel.

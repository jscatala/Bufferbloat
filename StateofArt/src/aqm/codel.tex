After several years using algorithms based on the average queue size, Nichols
and Van Jacobson realized that this approach although they have information of
congestion, no hard data on how serious is this. Jacobson et al, determined
that it may submit two types of queues: one that smooths bursts of arriving
packets turning it into a stable and continuous sent and tends to last less
than one RTT (called good queues); and those that only tend to create
excessive delays and lasting several RTTs (called bad queues). With this, they
were able to determine that the core of bufferbloat detection problem is
separating good from bad queues. Bearing in mind the above is that an
algorithm called \textit{Control delay} or CoDel by its acronym was
developed.

Presented in \cite{NicholsJacobsonCQD}, CoDel is based on the idea that it is
sufficient to maintain a threshold above which the tail can not exceed rather
than keeping a window of values to compute the minimum. Rather than measuring
queue size in bytes or packets, it is used the packet-sojourn time through the
queue. The time each packet spends in the queue is independent of the
transmission rate, the \gls{sojourn} is more valuable information about the behavior
of the buffer, in addition to being much more related to the effects the user may
experience. The sojourn is based on a timestamp corresponding to when the
packet arrives at the queue, at which this mark is added to the package
information. The minimum packet sojourn can be decreased only when a packet is
dequeued.

Another difference that occurs is that CoDel accepts the existence of queues,
but it is unacceptable to drop packets when you have a buffer with less than
\gls{MTU}'s bytes.

\begin{wrapfigure}{r}{0.5\textwidth}
    \centering
	$f(n) = \frac{100}{\sqrt{n}}*f(n-1)$ \\
	f(0) = 1
    \caption[CoDel droptime interval]{CoDel droptime interval, n=iteration}
    \label{fig:CoDeldroptime}
\end{wrapfigure}

But when the queue delay has exceeded target for at least interval, a packet
is dropped and a control law sets the next drop time accordingly to Figure
\ref{fig:CoDeldroptime}. The ways in which CoDel stops packet drop is either
when the delay queue is less than the target value or when the buffer contains
less than MTU's bytes. Furthermore the algorithm has additional logic that
controls reenter to dropping state too soon.


Beside the minimum packet sojourn, the other two parameters that CoDel needs
are the target and interval, which are almost self-explanatory. Target is the
acceptable standing queue delay and interval works as the  time on the order
of a worst-case RTT of connections through the bottleneck. The recommended
target value is 5ms since tests reveal that with this value, the utilization
of the links is the optimal. Interval is loosely related to RTT since it is
chosen to give endpoints time to react without being so long that response
time suffers. A setting of 100ms works well. \textit{CoDel's efficient
implementation and lack of configuration are unique features that make it
suitable for managing modern packet buffers}\cite{NicholsJacobsonCQD}.

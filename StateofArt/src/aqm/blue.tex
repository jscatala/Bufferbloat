\begin{wrapfigure}{r}{0.5\textwidth}
    %\rule{5.5cm}{7.1cm}
    \centering
	\begin{verbatim}
	Upon packet loss
	    if (now - last_update >freeze_t)
	        Pm = pm + d1
	        last_update = now
	Upon link idle
	    if (now - last_update >freeze_t)$
	        Pm = pm - d2
	        last_update = now
  	\end{verbatim}
    \caption{BLUE's algorithm}
    \label{fig:BLUEAlg}
\end{wrapfigure}

Introduced by Wu-chang Feng et al in \cite{FengBLUEAQM}, BLUE aims to manage
congestion by using packet loss and link utilization history. The concept
behind BLUE's development is to avoid drawbacks of RED like parameter tuning
problems or to determine the fluctuations of actual queue length. The key idea
is to perform queue management based directly on packet loss and link
utilization rather than on the instantaneous or average queue lengths.

Through the use of a single probabilistic indicator called $p_m$, it
simplifies the process of dropping or marking packets when they are enqueued.
Against a continuous dropping of packets due to overflow, the probability
increases, increasing the rate at which it sends back congestion notification.
Also, to control how this parameter changes over time, the
$freeze\_time$ is used to determine the minimum time interval between
updates.  The value on which to increase or decrease, if the link is idle, is
determined by $\delta_1$ and $\delta_2$.

The effects of BLUE, as proved in \cite{FengBLUEAQM}, are the reduction of the
buffer size, which also reduces the end-to-end delay and helps to improve the
effectiveness of the congestion control algorithm. BLUE has also shown that it
has a better performance when marking incoming packets, leading to congestion
notifications causes periods of sustained packet loss nor periods of continual
underutilization.

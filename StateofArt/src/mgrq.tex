As we have seen in previous section, the traditional way to manage router queue
is, after set a maximum length for the queue, accept packages until this length
is reached, and then drop subsequent packages until a packet from the stack as
been transmitted. This technique is called ``tail drop", and has served the
Internet well enough for years, but it has two important drawbacks:\\

\begin{itemize}
\item It allows, in some situations, a single or few flows to monopolize the
queue space, preventing other connections from getting room in the queue.
\item The signaling is produced only when the queue is full, so it allows queue
to maintain almost full status for long periods.
\end{itemize}

If queue is full or almost full, an arriving burst will cause multiple packets to
be dropped, an this behavior can produce a global synchronization of flows
throttling back followed by sustained period of lowered link utilization, which
will impact in a reduce of overall throughput, and if a long flows arrives in
that period, the lock-out of the queue.\\

Besides tail drop, another two techniques can be applied in these situations.
The ``random drop on full" will produce that the router drops a randomly selected
packet from the full queue, which requires an $O(N)$ walk through the queue, when a
new packet arrives. Under the ``drop front on full", the router drops the packet
at the front of the queue. Either of these solve the lock-out problem, but
neither solves the full-queue.\\

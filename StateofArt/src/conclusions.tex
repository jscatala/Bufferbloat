conclusions

http://lalith.in/2012/02/15/fun-with-tcp-cubic/
Examples of iperf and wireless 

\cite{MathisMacroCAA}

further

analysis in specific of how the three different flows (Elephant, mice and ant)\cite{HaElephants} \cite{evolvshortlongflows} reacts to the bufferbloat

How the linux kernel now reacts to bufferbloat. \cite{TokeLinux}

The effects under laying the use of CTCP\cite{Tan06compoundtcp}\cite{4146841} in Windows and the recommendations of the Bufferbloat community\cite{windowstips}

ver notas en \cite{MathisMacroCAA}

cerowrt project \cite{cerowrt}

Also from \cite{main:ref:1} and \cite{Vu-Brugier}, we can see that the rule-of-thumb doesn't longer apply to backbone routers, and a better estimator of the size of a buffer with n flows would be no more than $B = (\overline{RTT}xC)/\sqrt{n}$. With the assumption that short-flows plays a very small effect, and that the buffer size is dictated by the number of long flows, this factor will be proof that routers are much longer than they need to be, possible by two order of magnitude.
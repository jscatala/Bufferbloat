By analyzing all of the work done for this thesis, it is possible to conclude
that the objective of it was achieved considering the same frame of reference
as the criteria defined at the beginning of it being the \emph{objective of
checking the effects of Bufferbloat phenomenon, test the impact that it has in
on different networks and to propose solutions}. To accomplish
this, it first required to address the following general objectives:

\begin{itemize}
	\item To explain the \emph{Bufferbloat} phenomenon, and explain the impact that it could have over latency and throughput in Internet.
	\item To detect the presence by empirical measure of the latency and throughput in a TCP/IP Network.
	\item To propose possible solutions in the implementation of a network where the existence of excessively large and frequently buffers are detected.
\end{itemize}

Thanks to the various tests it was possible to demonstrate the presence and
feel the effects of \emph{Bufferbloat} in some networks. While the common
factor in them at first sight is their low bandwidth, you can not directly
assign this as a cause, as also contrary cases that had a bad performance
determined by physical effects such as routers were in poor condition or
environmental interference.

As proven in this thesis and corroborated in \cite{MathisMacroCAA}, in
general, a low latency network is wanted in order to exchange  messages between a server and a client. Since low latency make us feel
a faster web surfing and enables better performance in online games and VoIP
technology.

It is clear that the more far away a clients is from the server, the latency
will be higher, but what if we put more and more load to the network, how much
the latency can go up to? Having 12 times the latency when the network
overload is not normal and as mentioned by \emph{Jim Gettys} several times,
the culprit is Bufferbloat.

The effects on latency and navigation are catastrophic growing to full
charging time in minutes or even generate timeout sites. That if we take
today's world where more and more common to see applications that make use of
high bandwidth makes it almost impossible navigating multiple users with these
characteristics. Then, two users on the same network using such applications
should take turns to ensure a smooth operation.

Also from \cite{main:ref:1} and \cite{Vu-Brugier}, we can see that the rule-
of-thumb of sizing routers $B = (\overline{RTT}xC)$ doesn't longer apply.With
routes which have sizes much larger buffers than required, all it does is slow
down and hinder the passage of trading floors. The larger the buffer size, the
longer it takes for a packet to go through it, not adding any value in the
packet transfer and only adding additional latency.

In the past days networks were had much lower performance than they are today,
so as \emph{Bufferbloat} alike phenomena were more difficult to detect
because their effects were not as visible. Today with the advancement of
technology and the use of tools that feed us with information almost
instantaneous, any change or modification at any point within the
communication channel is evident, and why not say is magnified even to make
some systems collapse.

The complexity of deploying new algorithms into production environments is
such that no matter how many tests are performed over new algorithms, the
conditions under development environments will hardly match. Also do testing
at production environments are too hard and to complex\cite{Vu-Brugier} and if
they are not well designed can affects to users in a hardly way.

Thanks to Codel, and Bufferbloat community is possible to loom an effective
solution to the phenomenon of Bufferbloat. Although, even this phenomenon is
since a couple of years under study, the solutions are not yet fully proven
therefore may prevent from complete collapse, but these solutions are designed
for modern systems (eg. Codel requires BQL recommended for Linux kernel 3.5
onwards), so always will be some systems vulnerable to its effects. Apply
recommendations as QoS and bandwidth limiting for some applications are the
most common and easy to implement recommendations for any user, regardless of
the operating system on which they are working.

It is also important to keep a regular check of all the components in our
networks, not only to our computer. Effects of routers in evil, or using
technologies like wireless with the boom of recent years in homes fact that
interference between emitters is much more common than years ago, can be
determining factors in achieving this long-awaited win the last game shooter,
or close the business of our lives through a video conference.

Much still to be done in this field. As example, the analysis in specific of
how the three different flows (Elephant, mice and
ant)\cite{HaElephants}\cite{evolvshortlongflows} are affected to big buffers
and prioritize smaller flows can help to fix some of the issues like DNS
resolution, but how they react when the whole network is under the effects of
\emph{Bufferbloat}? The effects change if this occurs at a last-mile-router
or at a back-bone?

Thanks to studies as done by Hoiland-Jorgensen\cite{TokeLinux}, is that a
better understand of how the Linux Kernel reacts to \emph{Bufferbloat}, but
this follows up to all Linux based OS? With the advance of technology, new
devices are getting more cheap and most people can have access to them. Well
know is the case of Open-Wrt\cite{openwrt}, that is a Linux distribution for
embedded devices. Projects like Cero-WRT\cite{cerowrt}, can help to the
community to fight against \emph{Bufferbloat} in every piece that is
involved on the communication. But will this help in all devices? As example,
Arduino Yun is a microcontroller board, with an Atheros processor that
supports a Linux distribution based on OpenWrt named OpenWrt-Yun. The board
has built-in Ethernet and WiFi support. So, this board will well behave under
high consumption of bandwidth too? Also the effects of \emph{Bufferbloat}
under Windows machine are not well known. Beside the recommendations of the
Bufferbloat community\cite{windowstips} not much information can be found
about it. Study the effects under laying the use of
CTCP\cite{Tan06compoundtcp}\cite{4146841} will be a good point of start.


\begin{description}

\item [Bandwidth Delay Product (BDP)]:\hfill \\  An amount of data measured in
bits. It is equivalent to the maximum amount of data on the network circuit at
any given time. Commonly measured by the RTT*bandwidth.

\item[Bottleneck bandwidth]: \hfill \\  The smallest bandwidth along the path.
Packets cannot arrive at the destination any faster than the time it takes to
send a packet at the bottleneck rate.

\item [Bufferbloat]:\hfill \\  Refers to excess buffering inside a network,
resulting in high latency and reduced throughput.

\item [Congestion Window (cwin)]: \hfill \\  One of the factors that
determines the number of bytes that can be outstanding at any time. It
prevents the overload of the link with too much traffic. The size is
calculated by estimating how much congestion there is between the two places.

\item [Long Fat Network (LFN)]:\hfill \\ A network with a large bandwidth-
delay product. Often $>> 10^5$ bits (12500 bits). 

\item [Maximum segment size (mss)]: \hfill \\  Largest amount of data (in
octets) that a communication device can receive in a single TCP segment
(single IP datagram). Established by pass on the syn packet.

\item [Slow start threshold (ssthrsh)]: \hfill \\  It determine whether the
TCP should do Slow Start or Congestion Avoidance. It is initialized to a large
value, and after a congestion is signaled, cwin is divided in half and ssthrsh
is set to cwin.

\item [System Throughput]:\hfill \\ Corresponds to the fastest rate at which
the count of packets transmitted to the destination by the network is equal to
the number of packets sent into the network.

\end{description}

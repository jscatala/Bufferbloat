The tools that will be selected are determinates by the capability to
determinate the presece of the Bufferbloat phenomenon in network of study by the 
analysis of it indicator, the latency or the rount trip time. So, with this as 
main consideration for the tool's selection, the breanchmarking tools will be
choosen by the complexity and accuracy to measure and determine the RTT into a
IP/TCP connection, and the capability to replicate the results under similar 
contexts. Also, some other tools are selected that can help to determine the 
capability of the network to generate Bufferbloat.\\

The tools selected to be used in this tests are the following:

\begin{itemize}
    \item Speedtest test by Ookla
    \item Netalyzr by ICSI
    \item Iperf Tool with Tcptrace/Xplot.org
    \item Page Benchmarker extension for Google Chrome
    \item Smokeping Latency Tool
\end{itemize}

\subsubsection{Speedtest}

Developed by Ookla\footnote{\url{https://www.ookla.com/about}}, this tool is used
for most of the ISP's and many users in Chile to test their broaband's
connections globally. It can be used not only in their website 
\href{http://www.speedtest.net}{\textit{www.speedtest.net}} but also in Android,
iOS or Windows Phone. A command line interface developed in Python can be used 
too for testing internet bandwidth.\\

The server that will be selected to perform every test, either has or not the
fastest ping, is the one hosted by the Pontificia Universidad Cat\'olica de
Valpara\'iso (this host is the default selected most of times by the site also).  

\subsubsection{Netalyzer\cite{netalyzr}}

asdf

\subsubsection{Iperf}
Iperf is a well known and commonly used network testing tool. It can create a
TCP and UDP data streams and measure the bandwidth and the quality of a network
link. It can perform multiple tests like Latency, Jitter or Datagram Loss.\\

Iperf basically tries to send as much information down a connection as quickly
as possible reporting on the throughput achieved. This tool is especially useful
in determining the volume of data that links between two machines can supply.
This two machines define the network, one acting like a server and the second as
the client. For this scenario, the server will be the VPS that only will receive
the iperf connections (also will be running ssh but without further
interaction). The VM linux machine will work as or Iperf Client.\\  

As mentioned in iperf users mailing list \textit{
When one runs TCP tests, there are 2 things that block iperf from having clear
view of real throughput: buffering on sender's side (TCP/IP stack) and TCP
behaviour itself (acking). What iperf can measure is the pace with which it
sends data to TCP/IP stack; TPC/IP stack will only accept data from application
when buffers are not full. If the buffer is huge, iperf will see high
throughput initially, then it will drop. If there's congestion or retrasmission
going on, iperf will see it as lower throughput}\cite{iperfmaillist}, but the
data generated by iperf won't be further analized becouse the  idea behind using
this tool is a TCP's packet generator. This means that the packets generated by 
Iperf will captured and analized with tcpdump, tcptrace and xplot.org.\\

\subsubsection{Page Benchmarker}
Google Chrome extension

\subsubsection{Smokeping}
rtt test

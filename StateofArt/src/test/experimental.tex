The goal of the experiments outlined in this section is to examine different
residential and public networks with the objective to prove the existence of the
phenomenon under an uncontrolled scenario. This will be done by running a set of
tests with different tools, first to define and characterize the network, then
to measure and compare how the network behaves with and without load. More
specifically, the factor to be tested is the latency under load and analyzed to
identify if the latency that occurs is due to an excess of buffers or due to
some other problem.

This section aims to explain the setup used to perform the tests,
the tools selected for each of these tests and what is to be achieved and 
expected from each.

To test the existence of the Bufferbloat phenomenon, five tests are conducted 
described below. Each test will be repeated under the following contexts:

\begin{enumerate}
\item Twice on the same day in one network to determine if does exists a 
considerable variance in latency for different times of day.
\item Select different public and private networks with different 
\textit{``speeds''}.
\item Use the Ethernet cable in order to compare the results with those 
previously obtained using Wireless.
\end{enumerate}

Because it is intended to prove the existence under circumstances experienced by
everyday users, no kernel parameter will be amended nor changed, and only the 
ability to perform QOS on routers that have this feature is disabled. 
Furthermore, the overall question these tests seek to answer is the following:\\

\begin{theorem}
``The networks that we use every day, have the necessary to generate the 
Bufferbloat phenomenon whether under low loads and if does exists, the how 
seriours are the effects ?''
\end{theorem}

\subsubsection{Speed test}
The idea under this test, is to set a baseline by comparing the speed offered by
the ISP and the one that at the moment of test is serving. To find the speed 
provided, the tool used is the Speedtest in the web site of Ookla.\\


\subsubsection{Signs of trouble}
netalyzr tool\\

\subsubsection{Collapse test}
iperf to shoot\\

\subsubsection{Load benchmark test}
google chrome page benchmark.\\

\subsubsection{Smoke the path}
smokeping test.\\

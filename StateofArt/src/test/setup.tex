The tests were ran under a pseudo controlled environment, using one
physical  machine and a second virtual machine hosted on the first machine. These
systems runs under a regular OS without any modifications. Also, for some tests two other devices will be
added,  one acting as an Iperf Server and a regular Android Tablet that will
be used to add some extra load to the network when the Ethernet cable is used
as medium.

All of these tests will be carried out in a real-world scenario, where no
packet  prioritization is done by the server against our flows, the routes can
vary between each iteration of the same test, and many different flows will
collide with other flows from different sizes and types. Nor there is more
information about how the flows are treated by the queue manager algorithms or
about how they are  configured.

\subsubsection{Hardware Characterization}

\begin{description}

\item [Physical Machine]: \hfill \\
The physical machine runs as host OS, Windows 7 SP1, that works with an Intel(R)
Core(TM) i7-2670QM CPU \@ 2.20GHz with 8GB of available RAM. This machine will 
always be connected through its wireless adapter, a \textit{Broadcom Corp. 
BCM4313 802.11b/g/n Wireless LAN Controller (rev 01)}. The Ethernet controller 
is a \textit{Realtek Semiconductor Co., Ltd. RTL8111/8168 PCI Express Gigabit 
Ethernet controller (rev 06)} adapter, and will be bridged to the virtual 
machine for some tests.

\item[Virtual Machine]: \hfill \\
The virtual machine is hosted using VMWare Player 6.0.1, with 4 processors 
assigned for use plus 4GB of RAM, and with the Ethernet adapter connected only
for certain tests. The OS selected is a Debian based OS called Kali Linux, and
using the kernel release identified as Debian 3.12.6-2kali.

The wireless adapter is a AIR-802 USB adapter with Zydas chipset and a TP-link
8dbi antenna. This adapter is directly connected to the virtual machine and
hooked to the physical machine without the USB extension, this way any extra
signal  loss is avoided. 

\item[Iperf Server]: \hfill \\ 
This is a VPS hosted by Digital Ocean
\footnote{\url{https://digitalocean.com}} with 512MB Ram, 20GB SSD Disk, and
located in New York data center. This machine runs Ubuntu 12.04.3 x64 under
KVM software using as a processor an Intel Hex-Core 3 GHz.

\end{description}

The idea of using an external device to connect the virtual machine to the 
network and not using a bridged configuration provided by software, is mainly 
because with an USB device the machine will take care of all the management and 
administration of the device, avoiding any possibility that the host machine 
modifies or manages any flow. Also, by using a second machine to overload the 
uplink avoids overflow the queue on the machine that is performing 
the tests. With this, it is expected to minimize the possibility that our 
testing machine is causing extra latency, either by the saturation of the 
wireless channel, or by the queue in the traffic control subsystem into the 
kernel. 

For the tests, the Bufferbloat community\cite{bloat} has created a set of best 
practices\cite{tg12} to follow so the results are consistent and repeatable, 
but in this case, computers and routers will not be modified as it will attempt 
to analyze what an every-day-user experience. Those practices will be taken in 
consideration if it is the QoS present in routers and deactivate as recommended.
 

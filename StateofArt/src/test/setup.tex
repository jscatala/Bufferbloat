The tests will be ran under a pseudo controlled environment, using one physical 
machine and a second virtual machine hosted in first machine. These machines 
runs under a regular OS without any modification beside the ones that the 
operative system provides. Also, for some tests two other devices will be added, 
one acting as a Iperf Server and a regular Android Tablet that will be used to 
add some extra load to the network.\\

All of these tests will be carried out in a real-world scenario, where no packet 
priorization is done by the server against our flows, the routes can vary
between each iteration of the same test, and many different flows will collide
with other flows from different sizes and types. Nor is there more information 
about how the flows are treated by the QM algorithms or about how they are 
configured.\\

\subsubsection{Hardware Characterization}

\begin{description}

\item [Physical Machine] \hfill \\
The physical machine runs as host OS, Windows 7 SP1, that works with a Intel(R)
Core(TM) i7-2670QM CPU \@ 2.20GHz with 8Gb of avaible RAM. This machill will 
always be connected through its wireless adapter, a \textit{Broadcom Corp. 
BCM4313 802.11b/g/n Wireless LAN Controller (rev 01)}. The Ethernet controller 
is a \textit{Realtek Semiconductor Co., Ltd. RTL8111/8168 PCI Express Gigabit 
Ethernet controller (rev 06)}, adapter will be bridged to the virtual 
machine in some tests.\\

\item[Virtual Machine] \hfill \\
The virtual machine is hosted using VMWare Player 6.0.1, with 4 processors 
assigned for use plus 4Gb of RAM, and with the Ethernet adapter connected only
for certain tests. The OS selected is a Debian based OS called Kali Linux, and
using the kernel release identified as Debian 3.12.6-2kalii1.\\

The wireless adapter is a AIR-802 USB adapter with Zydas chipset and a TP-link 
8dbi antenna. This adapter is directly connected to the virtual machine and 
hooked to the physical machine without the adapter, this way any extra signal 
loss is avoided. \\

\item[Iperf Server] \hfill \\
This machine will be used as a server for the iperf test. This is a vps hosted 
by Digital Ocean \footnote{\url{https://digitalocean.com}} with 512MB Ram, and 
20GB SSD Disk, and located in New York data center. This machine runs Ubuntu 
12.04.3 x64 under KVM software using as a processor a Intel Hex-Core 3 GHz.

\end{description}

The idea of using an external device to connect the virtual machine to the 
network and not using a briged configuration provided by software, is mainly 
becouse with an USB device the machine will take care of all the management and 
administration of the device, avoiding any possibility that the host machine 
modify or manages any flow. Also, by using a second machine to overload the 
uplink, causes to avoid overflow the queue on the machine that is performing 
the tests. With this, it is expected to minimize the possibility that our 
testing machine is causing extra latency, either by the saturation of the 
wireless channel, or by the queue in the traffic control subsystem into the 
kernel). \\




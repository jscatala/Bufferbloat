The results are synthesized in order to analyze them comprehensively. 
The first test gave us the data to define the bandwidth of each
network with which subsequently went to work. Overall the upload link (uplink)
behaved as expected, performing in some cases gains of up to 50\% or more than
contracted. Obviously this could be due to generally low values ($\sim1Mbps$)
are therefore at a small variance is expressed as a high variation ratio. For
networks with greater uplink, between 1 and 5 Mbps also showed good
performance. On the other hand, the downlink reveal more variation between the
contracted and obtained. The circumstantial differences between a network with
good behavior to those with certain problems got reflected at this point. A
special case was the network \textit{polmos}. Strangelym after being tested with two
different Operating Systems (but on the same computer) results near
the contracted value were never obtained; not with other devices, where exhibit
values around the 40Mbps. The range values for the ping was not as expected,
but they still accepted values without being detrimental to smooth navigation
with a use that does not cause an excessive stress to the system.

With Netalyzr is possible to obtain a clearer idea of the fundamentals and the
inner workings of networks. Also most revealing information was obtained from
the theoretical perspective, the existence of Bufferbloat obtaining
approximate times of the duration of the buffers present in these networks.
This results was the most important for the continuity of the study, mainly
because it proceeded to rule out or find problems associated with other
factors, for example a defective router (as in the case of \textit{mbahamon})
or environmental interference.

Iperf revealed to us the behavior of the networks trying to use their full
capacity. Also checked the behavior of the protocols and the different phases
that they have (slow start and congestion avoidance periods). While this was
not within the objectives of the study, witness the behavior helped to verify
that the operating level protocols, were working as expected. Along with this,
it helped to ensure that the presence of buffers is lethal when trying to make
a fluid exchange data with a high level of network utilization.

The simulation of the experience gained by a user when trying to load a
website under periods of low and high loads also can enable to understand
better the effects of Bufferbloat. While the trend was already well defined by
the above tests, individual results included in the study were obtained, but
defined as outlayers, such as occurred in the network \textit{polmos\_w} where
a maximal value was obtained in the last iteration, which were expected  to be
more alike to the first iteration.

If it is believed that due to Bufferbloat the times was increased, thanks to
smokeping got demonstrated that not only increase considerably, but almost
impossible to web surf or almost very difficult to think in online games with
good performance. The resulting loss and increased latency are considerable
reaching about 12 times the initial value, and growing to 100\% loss
communication.

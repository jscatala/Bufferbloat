En los \'ultimos a\~nos, el crecimiento que ha experimentado el acceso y correspondiente uso de Internet ha sido abrumador. Dentro de las principalmente causas de este hecho se pueden mencionar el aumento del uso de Internet en nuestro diario vivir y la incorporaci\'on de Tel\'efonos Inteligentes y Tablets,  adem\'as del nuevo uso que se le esta dando a \textit{la red} gracias a la masificaci\'on y creaci\'on de nuevas aplicaciones tales como la transmisi\'on de videos y m\'usica en alta calidad, juegos en l\'inea y otras aplicaciones las que hacen un uso intensivo de la red.
\\

% what problem is your research trying to better understand or solve?
Con el paso del tiempo, el valor del hardware ha ido disminuyendo (\textit{Ley de Moore}) pero a la vez hemos ido creando la necesidad de tener m\'as y m\'as recursos a nivel de hardware; a\'un cuando los algoritmos que rigen las comunicaci\'ones a trav\'es de Internet no han variado en la misma proporci\'on. Lo anterior, sumado a la proliferaci\'on de \textit{buffers} de tama\~no excesivo est\'an siendo los causantes de un nuevo fen\'omeno en nuestras redes llamado \textit{``Bufferbloat''}.
\\

\textit{Bufferbloat} se le llama a la existencia de buffers excesivamente largos los cuales pasan la mayor parte del tiempo saturados, da\~nando o anulando los principales algoritmos que poseen los distintos protocolos de comunicaci\'on y que justamente tienen como tarea evitar la excesiva congesti\'on y posible pérdida de nuestros datos.
\\

% What is the scope of your study? a general problem or something specific?
Si bien el problema del \textit{``Bufferbloat''} se presenta bajo un entorno bastante acotado, ya que la teor\'ia dicta que es altamente probable que se presente en los routers que se ubican como puerta de entrada hacia el segmento de menor ancho de banda; este problema trae como consecuencia que toda la red se vea afectada por la disminuci\'on en el flujo de datos. 
\\

%What is your main clain or argument?
En el presente trabajo, se pretende demostrar a trav\'es del estudio del marco te\'orico sobre la formaci\'on y existencia del fen\'omeno del \textit{``Bufferbloat''} para el protocolo TCP, para luego con el uso de diferentes herramientas, probar emp\'iricamente a trav\'es de cinco diferentes pruebas en entornos no controlados, la presencia de este fen\'omeno en redes dom\'esticas de distintas caracter\'isticas.

According to the reports published by SUBTEL to the first half of 2013,
\textit{12.8} per 100 inhabitants have access to broadband
Internet\cite{OCDE}.  This steady growth is largely due to the emergence of
new applications such as streaming video and music, online gaming and
bandwidth intensive applications.

Not long ago, the links had a much more limited bandwidth than they have
today. With the evolution of electronics and telecommunications,
speeds have increased dramatically. With the current level of data flowing
through the network, it is important to control congestion that
might occur. This is one of the tasks of the routers.

It is important to remember that the traffic in a network is inherently
bursty, the role of the buffers in the router is to smooth the flow of
traffic. Without any buffering, to allocate the bandwidth evenly would be
impossible. But there are some problems with current algorithms; they use
tail-drop based queue management that has two big drawbacks: 1.- lockout 2.-
full queue that impact with a high queue delay.

Current low hardware prices make memory cheap,
and with the ``more is better'' mentality have led to the inflation and
proliferation of buffers everywhere\cite{NicholsJacobsonCQD}. When a router
joins two networks with different bandwidths, each packet is squeezed down in
bandwidth, it must stretch out in time since its size stays constant. When
the queue starts to grow, more and more memory is deployed leading to
massive standing queues. It turns out that this is a recipe for Bufferbloat.
Evidence of Bufferbloat has been accumulating over the past decade, but its
existence has not yet become a widespread cause for concern.

Bufferbloat creates large delays but no improvement in throughput. It is not a
phenomenon treated by queueing or traffic theory, which unfortunately results
in it being almost universally misclassified as congestion. The \textit{\gls{Bufferbloat}}
problem, making the window match the pipe size, is hard to address. Window
sizes are chosen by senders while queues manifest at bottleneck gateways.


As mentioned by Jim Gettys and Kathleen Nichols;
\textit{Today's networks are suffering from unnecessary latency and poor
system performance. The culprit is Bufferbloat, the existence of excessively
large and frequently full buffers inside the network. Large buffers have been
inserted all over the Internet without sufficient thought or testing. They
damage or defeat the fundamental congestion-avoidance algorithms of the
Internet's most common transport protocol. Long delays from Bufferbloat are
frequently attributed incorrectly to network congestion, and this
misinterpretation of the problem leads to the wrong solutions being
proposed}\cite{GettysNichols}.

The objective of this thesis work is checking the effects of
Bufferbloat phenomenon, test the impact that it has on different networks
and to propose solutions. To accomplish this, it first requires to
address the following general objectives:

\begin{itemize}
	\item To define the \textit{Bufferbloat} phenomenon, and explain the impact that it could have on latency and \gls{Throughput}(related to \gls{System Throughput}) in Internet.
	\item To detect its presence by measurements of the latency and throughput in a TCP/IP Network.
	\item To propose solutions in the implementation of a network where the existence of excessively large and frequently buffers are detected
\end{itemize}

In order to archive theses objectives:

\begin{itemize}
\item Develop appropriate tests to be able to prove the existence of \textit{Bufferbloat}
\item To test and differentiate the possible causes of the excessive latency and throughput reduction in a TCP/IP LAN and check how much is generated by \textit{Bufferbloat} or by a miss-configuration
\item To propose configuration of the TCP parameters in a Linux based machine or an algorithm that can help to minimize the phenomenon.
\end{itemize}

Chapter 2 explains the basis on which TCP was developed and
the fundamentals for the current operation of the Internet. The conservation
principle upon which all communication protocols are based will be explained.
It also describes each of the four phases of TCP.

In Chapter 3 we will review one of the most important components for the
communication and the main place where the packages are stored: the Routers.
We will analyze how their queues have a destructive size for
communication, why they are stalled at full capacity and cause excessive
latency. We also consider how to define the appropriate buffer size in a router.

Chapter 4 presented a technique developed to deal more efficiently with
actively congestion that could generate not only on endpoints but also on
routers. This technique is Active queue management (AQM). How the first
algorithms, such as RED and BLUE, were implemented and mention
CODEL which aims to solve the Bufferbloat phenomenon; This failures points,
in both TCP and AQM, are synthesized and analyze. All this will be presented in
Chapter 5.

Chapter 6 aims to define the methodology and tools that will be used to
perform each of the tests to determine the existence and the effects of
Bufferbloat. Each test is defined by its propose starting with some general
tests to check the behavior and sanity of the network and moving to a more
user-related effects of the Bufferbloat. All the tools used are Open Source
projects available for all commercial operative systems. In Chapter 7 the
results of all tests are presented, leading to the conclusions of this work in
chapter 8.

% 2.- Latency: When a new package arrives, it has to wait to the full queue get processed before its time, this way the package spent extra time waiting in the queue.
The latency a packet experiences in a network is made up of transmission
delay(the time it takes to send it across communications links), processing
delay(the time each network element spends handling the packet) and queuing
delay(the time spent waiting to be processed or retransmitted). But large
buffers only increases latency, and this only causes conflict with the needs
of nowadays applications.

Once packets in-fly reach a bottleneck, they begin to spooling. Because the
characteristics already explained, more and more packets are coming, and this
queue continues to increase, which leads that each new arriving packet spends
more time through the queue than the predecessor packet, which means an
increase in the latency. Then, eventually, packets start to be dropped,
notifying the hosts of the presence of congestion on the path.

As stated in \cite{Dischinger2007CRB}, it is quite common to find these high-
latency queues in the last mile. The causes can be many, but among the most
common are both link quality at homes, as different implementations of traffic
shaping methods implemented by ISPs, or massive buffers set by the latter to
avoid loss of data, which can add a delay of up to several milliseconds.
Bottlenecks at the Internet's edge can easily move between the wireless
access(when its bandwidth is slow) and the provider's up-link, both of which
can have highly variable bandwidths.

For networks that use coaxial cable, multiple clients concatenated their
upward flows in a single transmission, resulting in a burst with a large
volume of data, which can led to a high fluctuation in latency. This
concatenation can also generate jitter time, which can be produce a miss
interpretation for some protocols of incipient congestion and cause to enter
into congestion control avoidance too early. The other type of highly used
networks are the DSL, which the more distance is between the supplier reduces
its transmission rate. That is why it is needed advanced signal processing and
error correction algorithms which can lead to high packet propagation delays.

The main function of routers is to absorb bursts of traffic coming from the
routers, and ensure that links are used to their maximum capacity. A network
that does not have buffers, do not have the ability to store packets that are
waiting to be transmitted, so any package that is over the capacity of the
link will be dropped. In order to operate without buffers, arrivals must be
constant and predictable; synchronization of the entire network would be
necessary to avoid any loss. Such networks are complex in design, expensive to
implement and particularly restrictive.

The throughput of each network is limited by the capacity of the slowest link
causing a bottleneck; no matter how much more packets are introduced to the
network, the transfer will not be faster than that determined by this link.
That is why the major location of buffers in areas where bottlenecks occur is
essential for proper functioning of the network, but this transfer between
fast-to-slow networks is different on each route, even for the reverse path.

The presence of buffers is necessary to help to reduce data loss. Also because
in the past the high cost of memory kept the buffers quite small, causing it
to fill pretty quickly after the network became saturated, signaling to the
communications protocol the presence of congestion and thus the need for
compensating adjustments.

Unfortunately the network transport protocols, in order to operate at full
capacity, require that the hosts are notified in a timely manner when they
should back up and thus able to adapt its transmission rate to the available
capacity. The absence of such timely notification triggers the presence of
full buffers and increased communication latency.
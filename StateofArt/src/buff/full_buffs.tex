%	1.- Full buffers: Couse no signals buffers at the right time, the buffers are always full
Packet networks require absorb bursty arrivals that TCP performs in order to
communicate two hosts. When received, a packet is immediately validated, but
not necessarily immediately processed and transferred out without kept in a
buffer. That means that if the rate at which packets are received is greater
than delay that takes to process and remove a package from the buffer, it
tends to fill up\footnote{When this occurs, the receiving device may need to
adjust window size to prevent the buffer from being overloaded.} and stay
congested, contributing to excessive traffic delay and losing the ability to
perfom their intended function of absorbing bursts.

This standing queue is the essence of Bufferbloat and is the result of the
difference between the window and the bandwidth of the link, only creating
long delays in communication, but no benefit in overall throughput. \textit{It
is not a phenomenon threated by queue or traffic theory, which, unfortunately,
results in it being almost universally misclassified as congestion(a
completely different and much rared pathology)}\cite{CACMStaff}.

When the packet reach a bottleneck queue, is squeezed down in bandwidth but
must strech out in time since its size stays constant.

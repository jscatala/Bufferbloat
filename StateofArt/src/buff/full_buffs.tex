%	1.- Full buffers: Cause no signals buffers at the right time, the buffers are always full
Packet networks require to absorb bursty arrivals. When received, a packet is immediately validated, but
not necessarily immediately processed and transferred out without having spent some time in a
buffer. That means that if the rate at which packets are received is greater
than the delay to process and remove a package from the buffer, the buffer
fills up\footnote{When this occurs, the receiving device may need to
adjust window size to prevent the buffer from being overloaded.} and stays
congested, contributing to excessive traffic delay and losing the ability to
perform their intended function of absorbing bursts.

This standing queue is the essence of Bufferbloat and is the result of the
difference between the window and the bandwidth of the link, only creating
long delays in communication, but no benefit in overall throughput. \emph{It
is not a phenomenon treated by queue or traffic theory, which, unfortunately,
results in it being almost universally missclassified as congestion (a
completely different and much rarer pathology)}\cite{CACMStaff}.

When the packet reaches a bottleneck queue, it is squeezed down in bandwidth but
must stretch out in time since its size stays constant. This stretch is the
cause, at the buffer level, of the delay of the next packet in the queue.

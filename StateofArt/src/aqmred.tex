In the current Internet, dropped packets are used as a critical mechanism to
notify a end node when a congestion is presented. So, if a router is capable to
drop packets before the queue is full, and the end nodes take actions before the
buffers overflow, the full-queue problem is solved. This proactive approach is
know as ``active queue management" (AQM) and allows routers to control when and how
many packets to drop before buffers overflow.\\

For responsive flows, AQM can provide:
\begin{itemize}
\item reduce number of packets dropped in routers
\item provide lower-delay interactive service
\item avoid lock-out behavior
\end{itemize}

One AQM algorithm for routers is called ``Random Early Detection" or RED. The
algorithm drops arriving packets probabilistically, which increases as the
estimated average queue size grows, so it approach is based on the ``recent
past" events.\\

The RED algorithm consists of two main parts:

\begin{itemize}
\item Estimation of the average queue size
\item Packet drop decision
\end{itemize}

RED's particular algorithm for dropping is the culprit in the performance
improvement. 

TCP operates basically making two hosts exchange segments of data. The
connection between these two hosts is identified uniquely at each end by the
host's network address and a 16 bit port number. The communication between the
hosts is initiated by a three way handshake between them. This is where the
sequence number is synchronized between the participants. The sequence number
is a 32-bit number and it is the basis for a reliable data transport, working as
a mechanism to assure the exchange of data through an unreliable network
medium counting the bytes transmitted making possible be checked by the
receiver and be acknowledged. This is, starting from the initial sequence
number, each data byte sent as part of the connection has a corresponding
sequence number; and only after having being acknowledged by the receiver is
the data considered to be transmitted successfully.\\

TCP made use of the idea of pipe size and the knowledge there was reasonable
but not excessive buffering along the data path to send a window of packets at
a time. To control the amount of data that flows through the network path, the
receiver sent the information of how much data can receive at once, so the
network resource is utilized in a more efficiently way. This window of data
means how many bytes the receiver can work with, given the available buffering
and processing constraints. That is, window size represents how much data a
device can handle from its peer at one time before it is passed to the
application process. All the extra data in excess will be dropped. This window
is also a constrain to the sender, because the sender is not allowed to
transmits no more data than the window before waiting for the response of the
host as acknowledgment of the sent data.\\

To manage the data sent and the acknowledgment received for those packets, TCP
uses a cumulative scheme. After a packet is in flow from the server with its
corresponding sequence number, sent data goes to a retransmission queue where
it is held until the corresponding acknowledge from the other end has come in
or, to be resent if not acknowledgment within a timeout. When the
acknowledgment of a sequence number is received, the sender discard all data
with sequence numbers bellow the sequence number in the acknowledgment that
has arrive. For retransmission, TCP uses an adaptive scheme. The timeout is
automatically set from the measured round trip time of the connection,
taking into account the variance of the measured values\cite{JacobsonCAC}.
This helps avoid retransmitting potentially lost segments too quickly or too
slowly.\\

Because today's networks are dynamic and in different configurations, both
topologically speaking as a  bandwidth, TCP must handle these changes and
still be able to maintain communication between the two hosts. The lost of
communications can be an issue that TCP needs to take into account. In case of
lost of one packet means subsequent packets cannot be acknowledged until the
lost packet is retransmitted. This can lad to excessive retransmission and
unnecessary load. TCP extension has been developed that allows the receiver to
send selective acknowledgments of block of received data with sequence numbers
that are not cumulative with the data acknowledged in the traditional
way\cite{RFC2018}.\\

Since networks are shared and conditions change along the path, the algorithms
continually probe the network and adapt the number of packets in flight. It is
not hard also (but it is often the case) to find along the way decreases in
the bandwidth. This spots are the bottlenecks and they are important because
the performance or capacity of the entire connection (connection as a state
between the two hosts) is limited by the resources that this trace has. With
this, controlling the optimal rate of data transmission is a hard work, and
the receiver window as communicated by TCP is not a necessarily a very good
indicator. To fix this issue, TCP received an addition to its specification:
\textit{congestion window}. The congestion window plays a crucial role in
estimating the available bandwidth between the hosts. After this modification,
the minimum between the receiver window and the congestion window is used as
the transmit limit. All the TCP additions attempt to keep the network
operating near the inflection point where throughput is maximized, delay is
minimized, and little loss occurs.\\

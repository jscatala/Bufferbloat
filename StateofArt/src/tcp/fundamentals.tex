TCP operates basically making two hosts exchange segments of data. The
connection between these two hosts is identified uniquely by the network addresses and a 16 bit port number at both hosts. The communication between the
hosts is initiated by a three way handshake between them, where the
sequence number is synchronized between the participants. The sequence number
is a 32-bit number and it is the basis for reliable data transport through an unreliable network. This is, starting from the initial sequence
number, each data byte sent as part of the connection has a corresponding
sequence number; and only after having being acknowledged by the receiver is
the data considered to be transmitted successfully.

TCP makes use of the idea of pipe size and the assumption there was reasonable buffering along the data path to send a window of packets at
a time. To control the amount of data that flows through the network path, the
receiver sent the information of how much data it can receive at once, so the
network resource is used efficiently. Window size represents how much data a
device can handle from its peer at one time before it is passed to the
application process. All excess data will just be dropped. This window
is also a constraint to the sender, the sender is not allowed to
transmit more data than the window before acknowledgment of the data sent.

To manage the data sent and the acknowledgment received for those packets, TCP
uses a cumulative scheme. After a packet is in flight from the server with its
corresponding sequence number, sent data goes to a retransmission queue where
it is held until the corresponding acknowledgement from the other end has come in,
or to be resent if not acknowledged within a timeout. When the
acknowledgment of a sequence number is received, the sender discards all data
with sequence numbers bellow the sequence number in the acknowledgment that
has arrived. For retransmission, TCP uses an adaptive scheme. The timeout is
automatically set from the measured round trip time of the connection,
taking into account the variance of the measured values \cite{JacobsonCAC}.
This helps avoid retransmitting potentially lost segments too quickly or too
slowly.

Because today's networks are dynamic and in different configurations, both
topologically speaking as a  bandwidth, TCP must handle these changes and
still be able to maintain communication between the two hosts. In case of
loss of one packet means subsequent packets cannot be acknowledged until the
lost packet is retransmitted. This can lead to excessive retransmission and
unnecessary load. TCP extensions have been developed that allow the receiver to
send selective acknowledgments of block of received data with sequence numbers
that are not cumulative with the data acknowledged in the traditional
way \cite{RFC2018}.

Since networks are shared and conditions change along the path, the algorithms
continually probe the network and adapt the number of packets in flight. Often there are decreases in
the bandwidth along the way. Such spots are the bottlenecks\footnote{Or the \gls{BW}
points} and they are important because the performance or capacity of the
entire connection (connection as a state between the two hosts) is limited by
the resources that this trace has. With this, controlling the optimal rate of
data transmission is hard work, and the receiver window as communicated by
TCP is not a necessarily a very good indicator. To fix this issue, TCP
received an addition to its specification: \textit{congestion window}. The
congestion window plays a crucial role in estimating the available bandwidth
between the hosts. After this modification, the minimum between the receiver
window and the congestion window is used as the transmit limit. All the TCP
additions attempt to keep the network operating near the inflection point
where throughput is maximized, delay is minimized, and little loss occurs.

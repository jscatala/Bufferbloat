Because much of the basis for how TCP was designed in the early days of the
Internet, where speeds were low, the burden on the networks were only a few
kbps, and the networks interconnecting were separated by a couple
milliseconds, the use was ideal. But it is not hard to see that in the past
few decades, Internet growth has exceeded the expectations of even the most
experienced analysts and dreamers.

Intensive-consumption applications like on-line games, audio and video
streaming for music or on-line calls, torrent or P2P applications and other
sites that have a high level of resource utilization like Youtube, Facebook or
Netflix, have made networks of today are quite different from those that
formerly existed, but still running the same protocols\footnote{While many
have been updated, the basics remain the same}. If that was not enough, the
creation and incorporation of new devices such as cell phones, tablet and even
smart-watch with the ability to transmit data over the Internet, the more
necessary efficient and properly handling the wealth of information is
becoming

TCP relies on timely notification to adjust its transmission rate to the
available bandwidth, which is commonly signaled by the packet loss. But since
couple of years, and helped by lower prices on hardware manufacturing along
with the thought that more is always better, the manufacturers had decide to
prevent as much as possible the loss with the addition of larger buffers on
their devices. This simple action is slowly bringing a new collapse not only
TCP but data transmission in general, as can be seen in \cite{CACMStaff}

With the increase in the size of the buffers, the packages spent more time
``on the fly'' in this big buffers instead of dropped, signaling to TCP to
reduce the sending rate. When this large amount of data is dropped, causes TCP
sharply drop in transmission rate, freeing up bandwidth. Unfortunately, due to
the size of the buffers is static, when the new TCP slow-start phase start,
again due to the lack of signals out of time, TCP will work with diminish
performance. This problem is cyclical, resulting in exponential TCP back-off,
throughput degradation and very high latency

But the path on Internet are shared by multiple TCP streams so the buffers,
and this back-off behavior has a tendency to synchronize flows
\cite{main:ref:1}. This cause all the flows to throttle back their
transmission rates simultaneously, amplifying the effect. This decrease in
latency and reduced throughput are the effects of Bufferbloat phenomenon

Because much of the basis for how TCP was designed in the early days of the
Internet, where speeds were low, the burden on the networks were only a few
kbps, and the networks interconnecting were separated by a couple
milliseconds, the use was ideal. But it is not hard to see that in the past
few decades, Internet growth has exceeded the expectations of even the most
experienced analysts and dreamers.

Intensive-consumption applications like on-line games, audio and video
streaming for music or on-line calls, torrent or P2P applications and 
sites that have a high level of resource utilization like Youtube, Facebook or
Netflix, have made networks of today quite different from those that formerly
existed, but still running the same protocols\footnote{While many have been
updated, the basics remain the same}. If that was not enough, the creation and
incorporation of new devices such as smart phones, tablets and even smart-watches
with the ability to transmit data over the Internet, makes it more necessary to handle data traffic in an efficient way..

TCP relies on timely notification to adjust its transmission rate to the
available bandwidth, which is commonly signaled by packet loss. But since a
couple of years, and helped by lower prices of hardware along
with the thought that more is always better, the manufacturers have decided to
prevent the loss of data as much as possible with the addition of larger buffers on
their devices. This simple action is slowly bringing a new collapse not only
TCP but data transmission in general, as can be seen in \cite{CACMStaff}

With the increase in the size of buffers, the packages spent more time
``on the fly'' in big buffers instead of dropped, signaling to TCP to
reduce the sending rate. When finally a large amount of data is dropped,
it causes TCP to
sharply drop in transmission rate, freeing up bandwidth. Unfortunately, due to
the size of the buffers being static, when the new TCP slow-start phase starts,
again due to the lack of timely signals, TCP will work with reduced
performance. This problem is cyclical, resulting in exponential TCP back-off,
throughput degradation and very high latencies.

But the paths on Internet are shared by multiple TCP streams so the buffers,
and this back-off behavior, has a tendency to synchronize flows
\cite{main:ref:1}. This cause all the flows to throttle back their
transmission rates simultaneously, amplifying the effect. This decrease in
latency and reduced throughput are the effects of Bufferbloat phenomenon.

As stated in \cite{GettysNichols}, \textit{long delay from Bufferbloat are
frequently misattributed to insufficient bandwidth and this misinterpretation
of the problem leads to the wrong solutions being proposed}. As seen in the
past sections, the packet loss for TCP is not in itself a problem, conversely
it is essential for its functioning. The real problem is  the excessive and
consecutive packet losses that occurs when the buffers kept persistently
fulls. This effects may cause that modern applications timeout their
connections or in others may experience some delay. This is the real
effects and problems of Bufferbloat.

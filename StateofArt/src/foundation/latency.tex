When we move an amount of data, like a music file, it takes several minutes, or if we have lucky, several seconds. Smaller the file gets, less time is the duration of the transfer, but there is a limit. No matter how small the file becomes, we are stocked with a minimum time that we can never beat. That is call latency of the device. For an Ethernet network is 0.3\textit{ms}.\\

Maybe we don't notice the effect of this time, that when the amount of data is large enough, this time is too small compared with the time that takes the whole transfer. But, what happens with short-flows, like game streaming? Let's imagine that we want to stream audio over the net. 100\textit{ms} may not sound very much, but it's enough to notice a delay and echo in voice. A better case can be found in \cite{main:ref:3}, where the effects of latency in the transmissions and the buffers can be found.\\

There is no visible impact of varying the latency other than its direct effect of varying the bandwidth-delay product. Congestion can also be caused by deny of service attack that attempts to flood host or routers with large amount of network traffic. \\

